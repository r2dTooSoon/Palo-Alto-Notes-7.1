\section{Threat Landscape}
\subsection{Evolution of Malware}
\begin{itemize}
    \item \textbf{First Generation:}
        \begin{itemize}
            \item DoS Viruses (1986-1995)
            \item Macro Viruses (1995-2000)
        \end{itemize}
    \item \textbf{Second Generation:}
        \begin{itemize}
            \item Big Impact Worms (1999-2005)
            \item Malcode for Profit (2004-2010)
        \end{itemize}
    \item \textbf{Next Generation:}
        \begin{itemize}
            \item APTs for Profit
            \item Cybercrime
        \end{itemize}
\end{itemize}

\subsection{Advanced Persistent Threats}
\subsubsection{Introduction}
\begin{quote}
    An adversary that possesses sophisticated levels of expertise and significant resources, allowing them to create opportunities to achieve their objectives by using multiple attack vectors. 
    -National Institute of Standards and Technology (NIST)
\end{quote}

\subsubsection{Changes to the Attack Strategy}
\subparagraph{Time}
APTs unfold in multiple phases. Attackers must identify vulnerabilities to exploit the client, evaluate existing security controls, gain access to privileged hosts, find target data, and exfiltrate data.

\subparagraph{Sophistication}
Attackers are a sophisticated community of security researchers who constantly evolve their tools and methods to more effectively achieve their objectives.

\subparagraph{Directionality}
External threats quickly become internal attackers once a foothold is established.

\subparagraph{Evergreen}
Every-changing malware may evade antivirus programs. With polymorphic threats, the threat will mutate while keeping the original algorithm intact and avoid detection by antivirus software.

\subparagraph{Proactive}
Attackers generally have a clear objective and victim in mind prior to looking for entry into a targeted organization.

\textit{NOTE:} Zero-day or new malware can live on a network for days before being detected. 
Threat prevention tools require 72 hours on average to detect new malware. 
With WildFire, signatures are generated and delivered in as little as 5 minutes, thus reducing the time that a threat is effective.

\subsection{Attack Lifecycle}
\subsubsection{Introduction}
The \textit{Attack Lifecycle} is a sequence of events that an attacker goes through to successfully infiltrate a network and exfiltrate data from it.
The good news is that blocking just one stage in this lifecycle is all that is needed to protect a company's network and data from attack.

\subsubsection{Reconnaissance}
The collection of information about the target.
\begin{itemize}
    \item Discovering family information or hobbies for use in a phishing or watering hole attacks.
    \item Mapping networks
    \item Ping sweeps, fingerprinting, and port scanning.
\end{itemize}

\subsubsection{Weaponization}
In the \textit{Weaponization} phase, the attackers determine which methods to use.
They may choose to embed intruder code within seemingly innocuous files like a PDF or Word document or email message. Or, for highly targeted attacks, attackers may craft deliverables to match specific interests of an individual.

\subsubsection{Delivery}
The \textit{Delivery} phase typically consists of client exploitation. Methods commonly used to deliver a malicious payload include the following:
\begin{itemize}
    \item Phising email with a malicious attachment or link
    \item Malicious code injected into a compromised site known to be visited by a targeted victim.
    \item Attached exploit sent over SMTP.
\end{itemize}

\subsubsection{Exploitation}
The \textit{Exploitation} phase occurs once the attackers gain access into an organization. 
The attackers can activate attack code on the victim's host and ultimately take control of the target machine.

\subsubsection{Command and Control}
In the \textit{Command and Control (C2)} phase, attackers establish a command channel back through the internet to a specific server so that they can communicate and pass data back and forth between infected devices and their server.

\subsubsection{Installation}
In the \textit{Installation} phase, attackers will try to establish privileged operations, install root kits, escalate privileges, and persistence.
Attackers may have different motivations for exploiting the environment. Not all attack are for profit.
Attackers could try to exfiltrate data, destroy critical infrastructure, or deface web sites.

\subsubsection{Action on Objects}
The attacker performs the steps necessary to achieve his actual goal inside the victim's network.
This is an elaborate active attack process that takes thousands of small steps over several months.



\subsection{Data Breach and Attack Tactics}
\subsubsection{Target Data Breach}
\subparagraph{APTs in Action}
\begin{itemize}
    \item Recon on companies Target works with
    \item Spear phishing third-party HVAC contractor
    \item Breach Target network with stolen payment system credentials
    \item Moved laterally within Target network and installed POS malware
    \item Compromised internal server to collect customer data
    \item Exfiltrated data C2 servers over FTP
\end{itemize}

\subparagraph{Tools and Techniques}
\begin{itemize}
    \item \textbf{Recon}: Locate a vendor
    \item \textbf{Spear Phishing}: Obtain login credentials
    \item \textbf{BlackPOS}: Deliver POS malware
    \item \textbf{Botnet}: Internal C2
    \item \textbf{FTP}: Data exfiltration
\end{itemize}

\subsubsection{Aununak Infection}
\subparagraph{Methods of Infection}
\begin{itemize}
    \item Drive-by infections of Andromeda and Pony trojans through Neutrino Exploit Kits
    \item Spear phishing with malicious attachments on behalf of the Central Bank of the Russian Federation
    \item Identify already infected machines and purchase ownership from botnet herders.
    \item Money withdrawn directly from ATMs and cash sending channels like Yandex Money.
    \item These attackers gained access to 52 ATMs. Total amount of theft was over 1 billion rubles in Russia alone.
\end{itemize}

\subparagraph{Tools and Techniques}
\begin{itemize}
    \item \textbf{.doc .exe .cpl files}: Core malware payload
    \item \textbf{Mimikatz/Cain\&Abel}: Escalation of privileges
    \item \textbf{SoftPerfect}: Network scanning
    \item \textbf{SSHD backdoor}: C2 - Linux
    \item \textbf{Ammyy Admin}: Remote control - Windows
    \item \textbf{Team Viewer}: Remote control - Windows
\end{itemize}

\subsubsection{Desert Falcons Attack Tactics}
\subparagraph{Attack Tactics}
\begin{itemize}
    \item Use of social media chats deliver trojan files and encourage the victims to run them.
    \item Use of spear phishing emails that attempted to trick the victim into opening a malicious attachment.
    \item Use of commonly exploitable file types to extract, set up, and run the malware.
    \item A complete list of sensitive files is retrieved from the victim's machine.
    \item Depending on the importance of the victim, the surveillance is either intensified or dropped.
    \item More than 1 million files and documents containing sensitive information stolen from victims' computers and devices.
\end{itemize}

\subparagraph{Tools and Techniques}
\begin{itemize}
    \item \textbf{Facebook chat}: Malware delivery
    \item \textbf{Malicious posts}: Malware delivery
    \item \textbf{.lnk .scr .rar files}: Core malware payload
    \item \textbf{Password-protected archive}: Data exfiltration 
\end{itemize}

\subsubsection{Lessons Learned}
\begin{itemize}
    \item System hardening, integrity assurance, and software version/patch management are just the first steps in protecting against threats
    \item Inbound, outbound, and intra-tier network traffic needs to be controlled and monitored.
    \item Network segmentation is a positive step toward exposing only the minimal amount of information required for a specific organizational processes.
    \item True threat prevention that can detect previously unknown threats is a modern necessity.
    \item The security of an entire network can be negatively impacted by a single compromised asset.
\end{itemize}

\subsection{Threat Management Strategies}
\subsubsection{Integrated Approach to Threat Prevention}
Threat prevention capabilities of the Palo Alto Networks next-generation firewall include the following:
\begin{itemize}
    \item Application identification
    \item User identification
    \item URL filtering
    \item Vulnerability protection
    \item Anti-spyware
    \item Antivirus
    \item Traps
    \item File blocking
    \item WildFire advanced malware protection
    \item DoS protection
    \item Zone protection
\end{itemize}

\subsubsection{Strategy for Advanced Threat Prevention}
Advanced attacks are not random events. They are highly organized campaigns that leverage multiple techniques, operating simultaneously over long periods of time.
\newline
Five imperative strategies for advanced threat management are as follows:
\begin{itemize}
    \item Achieve complete visibility of all traffic, regardless of port, protocol, or encryption.
    \item Limit and control the traffic entering the network to reduce the amount of traffic that needs to be analyzed for zero days and APTs.
    \item Block known threats with signatures.
    Signatures for all types of malware are generated from millions of live virus samples sent to WildFire.
    \item Take automated action against unknown threats with WildFire. This sandbox offering analyzes suspicious samples to identify malicious behavior.
    \item Create closed-loop protections for all newly discovered threats and automatically update perimeter defenses against future attacks from the threat.
\end{itemize}

\subsection{Summary}
\begin{itemize}
    \item Threat Landscape
        \begin{itemize}
            \item Tools
            \item Techniques
        \end{itemize}
    \item Target, Anunak, Desert Falcon Attacks
        \begin{itemize}
            \item Data breach
            \item Attack tactics
        \end{itemize}
    \item Threat management strategies
        \begin{itemize}
            \item Integrated Approach
        \end{itemize}
\end{itemize}
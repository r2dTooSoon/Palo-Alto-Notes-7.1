%\documentclass{article}
%\usepackage[utf8]{inputenc}

%\title{Firewall 7.1: Install, Configure, and Manage}
%\author{das0}
%\date{October 2018}

%\begin{document}

\maketitle

\section{Platforms and Architecture}
\begin{itemize}
    \item Next-Gen Security Platform
        \begin{itemize}
            \item AutoFocus - Hosted service, Intelligence Cloud
            \item Aperture - Protects/Scans Dropbox et al.
            \item Traps - Prevent endpoint atacks from exe, data files, etc
        \end{itemize}
    \item Development of Unified Threat Management
        \begin{itemize}
            \item All-in-One - AV, URL Filter Appliance, Firewall, and IDS/IPS
        \end{itemize}
    \item Serial Processing in the UTM
        \begin{itemize}
            \item Needed new way to process data to reduce latency
        \end{itemize}
    \item Palo Alto Network Single-Pass Architecture
        \begin{itemize}
            \item Allows specific security policies based on identification of applications
            \item Uses packet inspection and library of application signatures
        \end{itemize}
    \item Palo Alto Network Firewall Architecture
        \begin{itemize}
            \item Control Plane
            \item Data Plane
        \end{itemize}
    \item Flow logic of the Next Gen Firewall
        \begin{itemize}
            \item Ingress
            \item Egress
        \end{itemize}
\end{itemize}


\section{Initial Configuration}

\subsection{Initial Access to the System}
    \begin{itemize}
        \item Configuring the Static MGT
            \begin{itemize}
                \item Connect via CLI; then set stat IP and commit
            \end{itemize}
        \item MGT Interface Stat Address Using WebUI
            \begin{itemize}
                \item 192.168.1.1
            \end{itemize}
        \item MGT Interface Dynamic Address
            \begin{itemize}
                \item Cloud/Multi-Tenant - DHCP
                \item IP often comes from a cloud orchestrator 
            \end{itemize}
    \end{itemize}

\subsection{Configuration Overview}
    \begin{itemize}
        \item Configure DNS and NTP SErvers
            \begin{itemize}
                \item Setup $\rightarrow$ Services
            \end{itemize}
        \item Configure Hostname and Domain
            \begin{itemize}
                \item General Settings
            \end{itemize}
        \item Banners and Messages
            \begin{itemize}
                \item Device $\rightarrow$ Setup $\rightarrow$ Management $\rightarrow$ Banners and Messages
            \end{itemize}
    \end{itemize}

\subsection{Configuration Management}
\subsubsection{Configuration Management}    
    \begin{itemize}
        \item Configuration Types
            \begin{itemize}
                \item After making changes click 'OK' and that becomes the Candidate; Click Commit to apply it to the configuration
                \item Logs = Monitor $\rightarrow$ Logs $\rightarrow$ Configuration
            \end{itemize}
        \item Commit Operation
            \begin{itemize}
                \item Device and Network
                \item Policy and Object
                \item Preview Changes
            \end{itemize}
        \item Transaction Locks
            \begin{itemize}
                \item Configuration Locks
                \item Commit Locks
            \end{itemize}
        \item Automatically Acquire Commit Lock
            \begin{itemize}
                \item Device $\rightarrow$ Setup for checkbox
            \end{itemize}
        \item Commit Queuing
            \begin{itemize}
                \item Commits must be executed one at a time
            \end{itemize}
        \item Configuration Management
            \begin{itemize}
                \item Save and rollback: Device $\rightarrow$ Setup $\rightarrow$ Operations
            \end{itemize}
        \item Configuration Management Auditing
            \begin{itemize}
                \item Shows changes in Red, Yellow, Green 
            \end{itemize}
    \end{itemize}

\subsection{Licensing and Software Updates}
\subsubsection{Licensing and Software Updates}
    \begin{itemize}
        \item Activate the firewall
            \begin{itemize}
                \item Serial Number required
            \end{itemize}
        \item Activate VM-Series
            \begin{itemize}
                \item Email authorization codes
            \end{itemize}
        \item Licensing
            \begin{itemize}
                \item Feature licensed separately
                \item Device $\rightarrow$ License
            \end{itemize}
        \item Dynamic Updates
            \begin{itemize}
                \item AV - daily
                \item Application and Threats - weekly
                \item BrightCloud URL Filter - daily
            \end{itemize}
        \item PAN-OS Software
            \begin{itemize}
                \item Check-now; Check release notes before implementation
            \end{itemize}
        \item Rapid Mass Deployment
            \begin{itemize}
                \item Bootstrap from USB
            \end{itemize}
        \item Power Operations
            \begin{itemize}
                \item Device $\rightarrow$ Setup $\rightarrow$ Operations
            \end{itemize}
    \end{itemize}

\subsection{Account Administration}
\subsubsection{Account Administration}
    \begin{itemize}
        \item Account Administration
            \begin{itemize}
                \item Initially only predefined admin has access
                \item User accounts can be tailored
            \end{itemize}
        \item Administrator Roles
            \begin{itemize}
                \item Defines role profiles
                \item CLI user rights defined by built-in roles
                    \begin{itemize}
                        \item None
                        \item Superuser
                        \item Superreader
                        \item deviceadmin
                        \item devicereader
                        \item vsysadmin
                        \item vsysreader
                    \end{itemize}
            \end{itemize}
        \item Using External AAA to Authenticate Admin Users
            \begin{itemize}
                \item Need server profiles to validate users
                \item RADIUS
                \item LDAP
                \item Kerberos
            \end{itemize}
        \item Authentication Profiles and Sequences
            \begin{itemize}
                \item Authentication profile
                \item Users checked against list
            \end{itemize}
        \item Creating Administrator Account
            \begin{itemize}
                \item Dynamic
                \item Role based
            \end{itemize}
    \end{itemize}
    
\subsubsection{Account Administrator (con't)}
    \begin{itemize}
        \item Global Authentication Settings
            \begin{itemize}
                \item Set up a RADIUS server 
                \item Configure RADIUS server and an Authentication Profile
                \item Add Authentication Profile
                \item There is no need to configure the admin user accounts on the firewall
            \end{itemize}
        \item Set Admin Passwords
            \begin{itemize}
                \item Device $\rightarrow$ Administrators
            \end{itemize}
        \item Minimum Password Complexity
            \begin{itemize}
                \item Prevent a common error and verify that the sum of the required characters is fewer than the max of 31 characters
                \end{itemize}
        \item Password Profiles
            \begin{itemize}
                \item Admin accounts can be have a password profile applied to alter security requirements
                \end{itemize}
        \item Reset to Factory Configuration
            \begin{itemize}
                \item With Admin User
                    \begin{itemize}
                        \item Erases all logs
                        \item Resets all settings
                        \item Saves a default config after the MGT IP is changed
                        \item Run command:     request system private-data-reset
                    \end{itemize}
                \item Without Admin User
                    \begin{itemize}
                        \item From the console port
                        \item Type maint during bootup
                        \item Choose Reset to Factory Default
                        \item Or load another config into memory
                    \end{itemize}
            \end{itemize}
    \end{itemize}
    
\subsection{Administrative Controls}
\subsubsection{Administrative Controls}
    \begin{itemize}
        \item Navigating the WebUI
            \begin{itemize}
                \item Most used
                \item 7 tabs across the top are Functional Tabs
                \item Left area is the menu based on tab selected
                \item Bulk page shows what work can be done
                \item Help button; search-able
            \end{itemize}
        \item Language Preference Setting
            \begin{itemize}
                \item Language selection is dynamic
            \end{itemize}
        \item WebUI Error Prompts
            \begin{itemize}
                \item Red underlines: Tabs that must be completed for a give interface
                \item Yellow highlights: Required fields
                \item OK button unavailable if interface is missing required info
            \end{itemize}
        \item CLI Modes
            \begin{itemize}
                \item Operational Mode $>$ prompt
                \item Configuration Mode $\#$ prompt
            \end{itemize}
        \item CLI Tools
            \begin{itemize}
                \item Tab and space autocomplete
                \item Linux shell
                \item ? or tab key
            \end{itemize}
        \item PAN-OS Rest API
            \begin{itemize}
                \item XML api
                \item requires api key
            \end{itemize}
        \item API Browser
            \begin{itemize}
                \item https://hostname/api
            \end{itemize}
    \end{itemize}
    
\section{Basic Interface Configuration}
\subsection{Security Zones}
\subsubsection{Security Zones}
    \begin{itemize}
        \item Introduction 
            \begin{itemize}
                \item Name
                \item Zone Type
                \item Interface
            \end{itemize}
        \item Security Zones and Policies
            \begin{itemize}
                \item Regulate and log traffic
                \item Zones
                    \begin{itemize}
                        \item DMZ
                        \item Internet
                        \item Guests
                        \item Data Center
                        \item Users
                    \end{itemize}
                \item Intrazone traffic is \textit{allowed} by default
                \item Interzone traffice is \textit{denied} by default
            \end{itemize}
        \item Security Zone Interfaces
            \begin{itemize}
                \item Interface can only have \textbf{one zone}
                \item Zones can have \textbf{multiple interfaces}
            \end{itemize}
    \end{itemize}
    
\subsection{Interface Types}
\subsubsection{Interface Types}
    \begin{itemize}
        \item Introduction
            \begin{itemize}
                \item Multiple interfaces can reside on one internet-facing device
            \end{itemize}
        \item Ethernet Interface Configuration
            \begin{itemize}
                \item Network $\rightarrow$ Interfaces $\rightarrow$ Ethernet
            \end{itemize}
        \item Flexible Deployment Options for Ethernet Interfaces
            \begin{itemize}
                \item TAP Mode
                \item Virtual Wire
                \item Layer 3
            \end{itemize}
        \item Ethernet TAP Mode
            \begin{itemize}
                \item Passively monitors network via SPAN
            \end{itemize}
        \item Configuring TAP Interfaces
            \begin{itemize}
                \item Select TAP from the interface listed
            \end{itemize}
    \end{itemize}

\subsection{Virtual Wire Interfaces}
    \begin{itemize}
        \item Introduction
            \begin{itemize}
                \item Bind two physical locations as "bump and wire"
                \item No VPN, switching, routing in VWire mode
            \end{itemize}
        \item Configuring a Virtual Wire Object
            \begin{itemize}
                \item Create object
                \item Can be tagged
            \end{itemize}
        \item Virtual Wire Subinterfaces
            \begin{itemize}
                \item Zone is required because traffic will flow between virtual wire interfaces
            \end{itemize}
        \item Configuring a Virtual Wire Subinterface
            \begin{itemize}
                \item Treated as a separate entity and can be assigned its own zone
            \end{itemize}
    \end{itemize}
    
\subsubsection{Layer 2 and Layer 3 Interfaces}
    \begin{itemize}
        \item Introduction
            \begin{itemize}
                \item Allow to forward traffic
            \end{itemize}
        \item Layer 2 Interface Example
            \begin{itemize}
                \item Useful for flat networks
            \end{itemize}
        \item VLAN Configuration
            \begin{itemize}
                \item Create VLAN object
                \item Then assign interface
            \end{itemize}
        \item Layer 2 Interface Configuration
            \begin{itemize}
                \item Select Layer 2 as type then add it to a VLAN and a security zone.
            \end{itemize}
        \item Layer 2 Subinterfaces
            \begin{itemize}
                \item Created like other subinterfaces
            \end{itemize}
    \end{itemize}
    
\subsubsection{Layer 2 and Layer 3 Interfaces (con't)}
    \begin{itemize}
        \item Configuring a Layer 3 Interface Overview
            \begin{itemize}
                \item Select Layer 3 and attach to a router
            \end{itemize}
        \item Configuring a Layer 3 Interface
            \begin{itemize}
                \item Need IP Address
                \item Zone
                \item Virtual Router
                \item DHCP and PPPoE
            \end{itemize}
        \item Assigning an IP Address to an Interface
            \begin{itemize}
                \item Support 
            \end{itemize}
        \item Configuring a Layer 3 Subinterface
            \begin{itemize}
                \item Have same controls as Layer 3 Interfaces.
            \end{itemize}
        \item Interface Management Profile
            \begin{itemize}
                \item Management traffic sent to or from the firewall through out-of-band MGT interface.
            \end{itemize}
    \end{itemize}
    
\subsubsection{Virtual Routers}
    \begin{itemize}
        \item Introduction
            \begin{itemize}
                \item Network $\rightarrow$ Virtual Routers
                \item Supports one or more static routes
                \item RIPv2
                \item OSPFv2
                \item OSPFv3
                \item BGPv4
            \end{itemize}
        \item Multiple Virtual Routers
            \begin{itemize}
                \item Dynmaic routes
                \item Static routes
                \item Must use CIDR
            \end{itemize}
        \item Virtual Router Static Routes
            \begin{itemize}
                \item Network $\rightarrow$ Virtual Routers $\rightarrow$ Static Routes $\rightarrow$ Add
                \item Must use CIDR
                \item 
            \end{itemize}
        \item Virtual Router Dynamic Routes
            \begin{itemize}
                \item RIP
                \item OSPF
                \item BGP
                \item PIM-SM
                \item PIM-SSM
                \item PIMv2
            \end{itemize}
    \end{itemize}

\subsubsection{Troubleshooting and Stats}
    \begin{itemize}
        \item Troubleshooting Routing
            \begin{itemize}
                \item Click \textit{More Runtime Stats}
            \end{itemize}
        \item More Runtime Stats
            \begin{itemize}
                \item To see routing table information through CLI, use the > \textit{show routing route}
            \end{itemize}
        \item Policy-Base Forwarding
            \begin{itemize}
                \item Destination IP address
                \item Source IP address
                \item Source zone
                \item Source user
                \item Destination application
                \item Destination
            \end{itemize}
    \end{itemize}

\subsubsection{IPv6 Capabilities}
    \begin{itemize}
        \item IPv6 Capabilities
            \begin{itemize}
                \item SLAAC
                \item LDAP
                \item RADIUS
            \end{itemize}
        \item Supported IPv6 Features
            \begin{itemize}
                \item Static Routing
                \item Dynamic Routing
                \item PBF
                \item NAT (NAT64 only)
                \item Site-to-Site VPN
                \item IPv6 over IPv4 IPSec Tunnel
                \item DNS Proxy
            \end{itemize}
        \item IPv6 Interface Configuration
            \begin{itemize}
                \item Supports dual stack for IPv4 and IPv6
            \end{itemize}
    \end{itemize}

\subsubsection{Interfaces and DHCP}
    \begin{itemize}
        \item DHCP Server
            \begin{itemize}
                \item The firewall can act as a DHCP server or a DHCP client
            \end{itemize}
        \item DHCP Server Options
            \begin{itemize}
                \item Lease duration
                \item gateway
                \item addresses of servers
            \end{itemize}
    \end{itemize}

\subsubsection{VLANs and Aggregate Interfaces}
    \begin{itemize}
        \item VLAN Interfaces
            \begin{itemize}
                \item All Layer 2 interfaces must be in Layer 2 zones
                \item VLAN interface must be in a Layer 3 zone
            \end{itemize}
        \item Configuring a VLAN Interface
            \begin{itemize}
                \item To enable hosts in the VLAN to reach destinations outside the VLAN network, configure the default gateway on the hosts to point at the VLAN interface.
            \end{itemize}
        \item Configuring Loopback Interfaces
            \begin{itemize}
                \item In-band management
                \item GlobalProtect Portal
                \item Gateway Functionality
                \item IPSec
            \end{itemize}
        \item Aggregate Interfaces
            \begin{itemize}
                \item Combine 8 ethernet interfaces
            \end{itemize}
        \item Create an Aggregate Interface
            \begin{itemize}
                \item Must all be the same type of interface (i.e., all copper or all fiber)
            \end{itemize}
        \item Assign an Interface to an Aggregate Group
            \begin{itemize}
                \item Create aggregate group and enable LACP and needed LACP settings.
            \end{itemize}
    \end{itemize}
    
    
    
    
    
    
    
    
    
    
    
    
    
    
    
    
    
    
    
    
    
%\end{document}
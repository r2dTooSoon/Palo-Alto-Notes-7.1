\section{Basic Content-ID}
\subsection{Security Profiles}
\subsubsection{Introduction}
Secuirty profiles are object that are added to security policies with the \textit{allow} action. Profiles are not necessary for security policies with the \textit{deny} action because no further processing is needed if the packet is to be dropped. 

\subsubsection{Antivirus Security Profile}
Defines actions to be taken if an infected file is detected as part of an application. If a virus is detected, the default action is to "reset-both" which means that the session will be dropped for an application using the UDP protocol. For apps using TCP the action resets the server and the client.
If the decoder is either IMAP or POP3 the default action is to \textit{alert}
If the decoder is SMTP the default action is also \textit{alert}, however an SMTP 541 error message is sent as part of the \textit{reset} action when a virus is detected.


\subsubsection{Antivirus Exception}
Defines a list of threats will be ignored by the antivirus profiles to reduce the number of false positive results or to ignore log messages irrelevant to the particular network 

\subsubsection{Anti-Spyware Security Profile}
	\begin{itemize}
		\item Default  - This profile is typically used for Proof of Concept
			\begin{itemize}
				\item Critical
				\item High
				\item Medium
			\end{itemize}	
		\item Strict - Profile applies the block response to all client and server
			\begin{itemize}
				\item Critical
				\item High
				\item Medium
			\end{itemize}		
	\end{itemize}		
These predefined profiles are \textit{read only} and cannot be modified or deleted. You can use these profiles as the baseline for any new security profile by using the \textit{clone} option

\subsubsection{Anti-Spyware Threat Types}
Can contain multiple rules to handle different types of threats. Each rule is configured with an action, a specific category of spyware to target, and severity levels.
\subsubsection{Anti-Spyware Exceptions}
Allows you to change the response to a specific signature.
Exceptions are made for individual signatures and can be restricted to specific IP address. 

\subsection{DNS Security}
\subsubsection{DNS Sinkhole}
Queries involve forging responses to select DNS queries so that clients on the network connect to a specified host rather than to the malicious system pointed to by DNS.

\subsubsection{DNS Sinkhole (Cont.)}
The infected hosts are easily identified in the traffic logs or using reports.
By  default, the Sinkhole address is set to the local loopback address

\subsubsection{DNS Signatures}
Objects $\rightarrow$ Security Profiles $\rightarrow$ Anti-Spyware
External dynamic list:
	\begin{itemize}
		\item Hosted text file
		\item Dynamic list
		\item Separate action for each list
	\end{itemize}
Threat ID:
	\begin{itemize}
		\item DNS signature exceptions
	\end{itemize}
Passive DNS monitoring:
	\begin{itemize}
		\item Firewall collects DNS information
		\item Forwards to Palo Alto Networks for analysis
	\end{itemize}


\subsection{Vulnerability Protection}
\subsubsection{Vulnerability Protection Security Profile}
A Security policy can include specification of a vulnerability protection profile that determines the level of protection against buffer overflows, illegal code execution, and other attempts to exploit system vulnerabilities.
	\begin{itemize}
		\item Default - Applies the default action to all client and server vulnerability protection events. This profile is typically used for Proof of Concept or first-phase deployments

		\item Strict - The profile applies the block response to all client and server vulnerability protections events and uses the Default action for low and info vuln events.
	\end{itemize}
Predefined profiles are read-only and cannot be modified or deleted

\subsubsection{Creating a New Vulnerability Protection Profile}
Customized profiles can be used to focus vulnerability checking for specific threats and types of traffic
	\begin{itemize}
		\item \textbf{Allow}: Threats are allowed to pass with no further actions
		\item \textbf{Alert}: Threats are allowed to pass and are logged in the Threat log.
		\item \textbf{Block}: Threats are blocked by the firewll and are logged in the threat log.
	\end{itemize}

\subsubsection{Vulnerability Exceptions}
Allows for the change of response to a specific signature. 
Exceptions are made for individual signatures and can be restricted to specific IP addresses.
IP addresses must be entered as unicast addresses.

\subsubsection{Threat Log}
Records each security alarm generated by the firewall. 
Each entry includes the date and time, the threat type, the to and from zones, the IP addresses of the victim and attacker, the ports, application name, the action, and severity.

\subsubsection{Creating an IP Exemption from the Treat Log}
The need for exception to the vulnerability and anti-spyware profiles often is not known until a user complains about lost functionality.


\subsection{URL Categories and Filters}
\subsubsection{URL Filtering Security Profile}
URL Filtering is a subscription product. Enables you to monitor and control how users access the web over \textit{HTTP} and \textit{HTTPS}.
The firewall comes with a default profile that is configured to block websites such as known malware site, phishing sites, and adult content sites.

You can download predefined sets of web categories from Palo Alto Networks. 
	\begin{itemize}
		\item PAN-DB (default)
		\item BrightCloud
	\end{itemize}

\subsubsection{URL Category vs. URL Filtering Security Profile}
URL Category	
	\begin{itemize}
		\item Used as a match condition in policies
		\item Matches only predefined or custom categories
		\item Action is determined by policy
		\item Logged as part of the entry for a policy in the traffic log
	\end{itemize}
URL Filtering Security Profile
	\begin{itemize}
		\item Applied to traffic allowed by Security policy
		\item Matches predefined or custom categories, as well Block/Allow Lists
		\item Action can be configured differently for individual categories or URLs
		\item Logged in the URL Filtering log
	\end{itemize}
	
\subsubsection{URL Filtering Sequence}
Order processed:
    \begin{enumerate}
        \item Block list
        \item Allow list
        \item Custom categories
        \item URL categories:
            \begin{itemize}
                \item PAN-DB
                \item BrightCloud
            \end{itemize}
    \end{enumerate}

\subsubsection{Custom URL Categories}
Each custom category can be controlled independently and has an action associated with it in each URL Filtering profile
    \begin{itemize}
        \item Alert
        \item Allow
        \item Block
        \item Continue
        \item Override
    \end{itemize}
\subsubsection{URL Filtering Actions}
The available actions are:
    \begin{itemize}
        \item \textbf{Allow}: Allows the user to access the website; no log or user message is generated
        \item \textbf{Alert}: Allows the user to access the website but adds an alert to the URL log
        \item \textbf{Block}: Traffic is blocked, a block log entry generated, and a response page is sent to the user's browser
        \item \textbf{Continue}: Sends a response page that prompts the user to click continue to proceed, and logs the action
        \item \textbf{Override}: Sends a response page and allows the user to access the blocked page after entering a password, and logs the action. \textit{Device $\rightarrow$ Setup $\rightarrow$ Content ID $\rightarrow$ URL Admin Override}
    \end{itemize}

\subsubsection{URL Filtering Response Pages}
\paragraph{Block pages}are displayed when a user attempts to access a URL or URL category with a configured action of
Block, Continue, or Override.

\paragraph{URL Filtering}response pages require the configuration of a Layer 3 interface on the firewall with an Interface Management profile. 

\subsection{URL Categories and Filters (con't)}
\subsubsection{Configuring the URL Admin Override Password}
URL Filtering Profile
\begin{itemize}
    \item The \textit{override} action requires the user to enter a password rather than to simply click a button, as the Continue action does.
    \item A firewall can have only one URL Admin Override password.
\end{itemize}
Mode setting determines whether the block page is delivered transparently or by a redirect.
\begin{itemize}
    \item Select \textit{Redirect} and enter the IP address for redirection
    \item The IP address must correspond to a Layer 3 interface on the firewall, with an \textit{Interface Management Profile} assigned with the \textit{Response Pages option} enabled
\end{itemize}

\subsubsection{Search Engine Cached Sites}
\textbf{Cache filtering} is enabled by default. Updates are provided through URL Filtering dynamic content updates. Cache filtering uses a secondary, recursive lookup, with each individual lookup following the same URL categorization flow.
\subsubsection{Safe Search Enforcement}
URL Filtering profile and is used to prevent users who are searching the internet using one of the top three search providers from viewing search results unless the strict safe search option is set in the browser or user account:
Safe search is a best-effort setting. The safe search setting can be set in one or both of the following ways:
\begin{itemize}
    \item Account setting
    \item Browser setting
\end{itemize}
 If a user is logged in to a search provider site the connection is over SSL, so decryption must be enabled on the firewall for the Safe Search Enforcement to function.

\subsubsection{URL Filtering Log Container Page}
A Container Page is the main page that a user accesses when visiting a website, but additional websites may be loaded within the main page.

\subsubsection{URL Filtering Log}
Contain log entries for the following actions:
\begin{itemize}
    \item Alert
    \item Continue
    \item Override
    \item Block
\end{itemize}

\subsection{Security Profile Administration}
\subsubsection{Security: PAN-DB vs BrightCloud}
Two URL Filtering Services are supported but only one at a time can be applied.
\paragraph{PAN-DB}
URL Filtering license is required. Two Offerings:
    \begin{itemize}
        \item Public Online
        \item Private Offline
    \end{itemize}
Uses a seed database for initial configuration, then the device stays in sync with cloud servers.
Attempts lookups from:
\begin{itemize}
    \item Caches
    \item PAN-DB cloud servers or an M-500 appliance (offline)
\end{itemize}

\paragraph{BrightCloud}
URL filtering license required. Single offering:
\begin{itemize}
    \item Public Online only
\end{itemize}
Relies on a URL database file, which is saved to disk and updated daily.
Attempts lookups from:
\begin{itemize}
    \item Caches
    \item Disk database file
    \item Cloud servers
\end{itemize}

\subsubsection{URL Filtering Cache: PAN-DB}
\textbf{Dynamic URL Filtering} is enabled by default and is not configurable if the firewall is using PAN-DB.
\textbf{Management Plane Cache} is initially created from a seed database file downloaded from the cloud server.

\subsubsection{PAN-DB Private Offline Database Server}
On-premises solution that is suitable for organizations that prohibit or restrict the use of the PAN-DB public cloud service.
M-500 can operate in three modes; however, it can serve only one of the three modes:
    \begin{itemize}
        \item Panorama mode
        \item Logger mode
        \item PAN-URL-DB mode (private cloud, or offline mode):
            \begin{itemize}
                \item No web interface; only CLI
                \item Cannot be managed by Panorama
                \item Cannot be deployed in a high-availability pair
            \end{itemize}
    \end{itemize}
\subsubsection{URL Filtering Updates:  BrightCloud}
Updated daily and includes a database of about 20 million websites.

\subsubsection{Recategorization Requests}
Can submit recategorization forms to respective vendors.

\subsubsection{Enabling Security Profile Groups}
\textbf{Security profiles} are enabled on individual Security policies. Groups are recommended for sets of profiles that are commonly assigned together.

\subsubsection{Zone Protection Profile}
\subsubsection{Introduction}
\begin{itemize}
    \item Zone Protection profiles are set on specific zones and address issues with Layer 3 and 4 protocol-based attacks
    \item Multiple Zone Protection profiles can be created on the PAN device, but a zone can have only a single profile applied to it.
    \item All traffic received on any interface in that zone is examined based on the Protection Profile when the zone is the session destination zone.
    \item All flood protections are configured to protect \textit{SYN flood}, \textit{UDP flood}, and \textit{ICMP flood}.
    \item The value set in the \textit{Alert}, \textit{Activate}, and \textit{Maximum} fields is the packets per second from one or many hosts to one or many destinations in the zone.
    \item Packets to the destination zone are sampled at an interval of one second, to determine if the rate matches the threshold.
\end{itemize}
Zone protection is packet-based and uses the header information to maintain a counter and to check TCP flags; it does \textit{not} provide signature-based prevention.
\subsubsection{Zone Protection Types}
\textbf{Reconnaissance Protection} is used to prevent or alert administrators on attempts such as
\begin{itemize}
    \item Port scans
    \item ICMP sweep
\end{itemize}
Packet-based attacks use malformed traffic to adversely affect target systems
\subsubsection{Enabling Zone Protection}
Profiles are enabled on a per-zone basis. Each zone can have only one Zone Protection profile assigned to it.
Zone protection applies only when the zone is used as the destination zone of a session.


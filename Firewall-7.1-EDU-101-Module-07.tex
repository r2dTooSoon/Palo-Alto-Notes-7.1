\section{File Blocking and WildFire}
\subsection{File Blocking Concepts}
\subsubsection{Introduction}
\textbf{File blocking} enables the blocking of prohibited, malicious, and suspect files from being downloaded - and/or uploaded - to protect users, their devices, and of course the corporate network.

\subsubsection{File-Blocking Configuration}
The available actions for File Blocking profiles
    \begin{itemize}
        \item \textbf{Alert}: Allows the file transfer to continue but adds an alert to the URL log
        \item \textbf{Block}: Traffic is blocked, a block log entry is generated, and a response page is sent to the user's browser if the traffic is web-based.
        \item \textbf{Continue}: Sends a response page requiring the user to click Continue to proceed, and will log the action
    \end{itemize}

\subsubsection{File-Block Configuration: Multilevel Encoding}
By default, four levels of decoding protection are in PAN-OS 7.0
Create the File Blocking profile set the file type to \textit{Multi-Level-Encoding} with the action set to \textit{Block}
Supported encoding methods include the following:
    \begin{itemize}
        \item Gzip
        \item Pkzip
        \item Chunked encoding
        \item Base64 encoding
        \item Uuencode
        \item Qpencode
    \end{itemize}
\subsubsection{Drive-By Download Protection}
The File Blocking Profile looks within the application session, sees that a download is taking place, and verifies with the user if the file is an approved download.
\subsubsection{Data Filtering Log}
Allows administrators to monitor firewall interactions with WildFire.
    \begin{itemize}
        \item \textbf{wildfire-upload-success}: the file was actually sent to the cloud;
        this means the file is not singed by a trusted file signer and it has not yet been seen by WildFire
        \item \textbf{wildfire-upload-skip}: the file was not uploaded because it was seen by WildFire before,
        and it was determined to be malware. This will generate a WildFire report.
    \end{itemize}

\subsection{WildFire Platform}
\subsubsection{Introduction}
    \begin{itemize}
        \item Ready for front-line deployment
        \item True enforcement
        \item Superior analysis
        \item Scalability
        \item Three solutions:
            \begin{itemize}
                \item WildFire Public Cloud
                \item WF-500 Private Appliance
                \item WildFire Hybrid Cloud
            \end{itemize}
    \end{itemize}
\subsubsection{WildFire Cloud}
Makes use of your on-premises firewalls in conjunction with the Palo Alto Networks cloud-based analysis engine to deliver an ideal blend of protection and performance. 

\subsubsection{Signature Generation Capabilities on the WF-500 Appliance}
\begin{itemize}
    \item Generate local malware and command and control (C2) signatures directly on the WildFire appliance
    \item Provides three types of protection
        \begin{itemize}
            \item \textbf{Antivirus signatures}: Prevent malware downloads
            \item \textbf{DNS signatures}: Block C2 traffic
            \item \textbf{URL malware categorization}: Block C2 traffic
        \end{itemize}
    \item Distribute local WF-500 signatures to all PAN-OS firewalls across the network for consistent network protection
    \item Signatures updated every 5 minutes
\end{itemize}

\subsubsection{WildFire Subscription Service}
Provides enhanced services for organizations that require immediate coverage for threats, frequent WildFire signature updates, advance file type forwarding, and the ability to upload using the WildFire API.

\subsubsection{Identify and Protect Against Malicious Email Links}
When a malicious link is detected, a log is generated for the malicious email that indicates the sender, receiver, and subject for quick identification.

\subsubsection{Email Header Information}
The firewall captures email header information that includes the sender, the recipient(s), and the subject fields,
This header information is sent to WildFire along with the corresponding email attachment(s) and any link(s) within the email. 

\subsubsection{Configuring WildFire}
\textbf{Device $\rightarrow$ Setup}
Default file sizes are:
    \begin{itemize}
        \item flash = 5MB
        \item apk = 10MB
        \item pdf = 200KB
        \item jar = 1MB
        \item pe = 2MB
        \item ms-office = 500KB
    \end{itemize}

\subsubsection{Security Profile: WildFire Analysis}
\textbf{Objects $\rightarrow$ Security Profile $\rightarrow$ WildFire Analysis}
\newline
Allows for the configuration of where files will be sent.

\subsubsection{Security Profile: WildFire Analysis (con't)}
    \begin{itemize}
        \item Create the new WildFire Analysis Profile object and set the file types, applications, direction, and analysis location
        \item Attach this new WildFire Analysis Profile object to a Security policy rule.
    \end{itemize}

\subsubsection{WildFire Dashboard}
The portal opens to display the dashboard, which lists summary report information for all of the firewalls associated with specific WildFire account or Support account.
The display includes the number of analyzed files and indicates how many are infected with malware, are benign, or are pending analysis.

\subsubsection{WildFire Dashboard Reports}
To display the list of available reports, click \textit{Reports} button.

\subsubsection{WildFire Submissions Log}
\textbf{Monitor $\rightarrow$ Logs $\rightarrow$ WildFire Submissions}

\subsection{WildFire Analysis}
\subsubsection{WildFire Analysis Verdicts}
    \begin{itemize}
        \item Benign
            \begin{itemize}
                \item Files categorized as benign are safe and do not exhibit malicious behavior
            \end{itemize}
        \item Grayware
            \begin{itemize}
                \item Files categorized as grayware do not pose a direct security threat, but might display otherwise obtrusive behavior
                \item Grayware can include
                    \begin{itemize}
                        \item adware
                        \item spyware
                        \item browser helper objects
                    \end{itemize}
            \end{itemize}
        \item Malware
            \begin{itemize}
                \item Files categorized as malware are malicious in intent or nature and can pose a security threat.
                \item Malware can include:
                    \begin{itemize}
                        \item viruses
                        \item worms
                        \item trojans
                        \item remote access tools
                        \item rootkits
                        \item botnets
                    \end{itemize}
            \end{itemize}
    \end{itemize}


\subsubsection{WildFire Analysis Verdict Example}
The public cloud and the WildFire appliance are supported, and all firewall platforms are supported.
When files are identified as malware, a signature is generated and distributed by the WildFire cloud to prevent future exposure.

\subsubsection{Report Incorrect Verdict}
If you think that a file was incorrectly categoriezed, you can report it from within PAN-OS software.

\subsubsection{WildFire Portal: Incorrect Verdict}
WildFire reports indicate whether a file was benign, grayware, or malware. As an administrator, you can resubmit a file if you think that this determination is incorrect. 
Find the link at the bottom of the report itself from within the portal or from the WildFire Analysis report in PAN-OS software.

\subsubsection{WildFire Settings}
The WildFire grayware analysis result was introduced to clearly identify executables that behave similarly to malware but are not malicious in nature or intent.
The grayware verdict allows the security responder to quickly distinguish malicious files on the network from grayware and to prioritize accordingly.
Though antivirus signatures are not generated for grayware, WildFire logs can continue to alert the security responder to endpoints that download grayware to assess if such events are of concern.

\subsubsection{Log Forwarding of WildFire Verdicts}
Forward WildFire verdicts to Panorama or to other external systems or services.
\section{Basic-User-ID}
\subsection{User-ID Overview}
\subsubsection{Introduction}
The ultimate goal of User-ID is to allow the ability to write policy and view logs with information about users and applications instead of using IP addresses and ports
\begin{itemize}
    \item Identify the user by username and user group instead of by IP address
    \item Create policies and reports based on users and groups
\end{itemize}

\subsubsection{Where is User-ID Used?}
\begin{itemize}
    \item Within the logs and reports:
        \begin{itemize}
            \item Sort and filter logs by user/groups
            \item Monitor $\rightarrow$ Traffic
        \end{itemize}
    \item Within the Security policy:
        \begin{itemize}
            \item Control application use by group and user
            \item Separate unknown user traffic from known user traffic
            \item Policies $\rightarrow$ Security
        \end{itemize}
\end{itemize}

\subsubsection{User-Base Policies}
\begin{itemize}
    \item Policies can be applied to users and groups regardless of the IP address
    \item User policies:
        \begin{itemize}
            \item Security
            \item QoS
            \item Policy Based Forwarding
            \item Decryption
            \item DoS protection
        \end{itemize}
\end{itemize}

\subsubsection{Group and User Mapping}
Group $\leftrightarrow$ Username $\leftrightarrow$ IP Address
\begin{itemize}
    \item Group mapping:
        \begin{itemize}
            \item Gather a list of all available users and their groups from the LDAP Server
        \end{itemize}
    \item User mapping:
        \begin{itemize}
            \item Map the IP address of a traffic to a username using a User-ID agent and other mechanisms.
        \end{itemize}
\end{itemize}

\subsubsection{Group Mapping}
\begin{itemize}
    \item The Palo Alto Networks firewall must retrieve the list of groups and users from the directory server:
        \begin{itemize}
            \item Find specific users
            \item Find groups and group membership
            \item Maintain user-to-group mapping
        \end{itemize}
    \item Many LDAP directory servers are supported:
        \begin{itemize}
            \item Microsoft Active Directory (AD)
            \item Novell eDirectory
            \item Sun ONE Directory Server
        \end{itemize}
\end{itemize}

\subsubsection{User-ID Methods}
A combination of methods is used to find user and group information and to map those users to session source IP address(es):
\begin{itemize}
    \item \textbf{Server monitoring}: User-ID agent monitors Exchange Servers, domain controllers, or Novell eDirectory servers.
    \item \textbf{Client probing}: User-ID agent probes Windows clients using Windows Management Instrumentation (WMI)
    \item \textbf{Syslog}: User-ID agent listens for auth syslog messages from network access control systems, wireless controllers, 802.1x devices, Apple Open Directory servers, and proxy servers
    \item \textbf{Terminal Services agent}: Port mapping for Windows/Citrix terminal servers
    \item \textbf{Captive Portal}:
        \begin{itemize}
            \item If User-Agent can't be used
            \item Web traffic authenticated using NTLM or AAA form
        \end{itemize}
    \item \textbf{GlobalProtect}: Login credentials for client VPN
    \item \textbf{User-ID XML API:}
        \begin{itemize}
            \item When other methods can't be used
            \item Captures login events and sends them to the User-ID agent or to the firewall
        \end{itemize}
\end{itemize}

\subsubsection{User Mapping with User-ID Agent}
\begin{itemize}
    \item Exchange servers, domain controllers, or eDirectory servers
        \begin{itemize}
            \item User-ID agent: server monitoring
        \end{itemize}
    \item Windows clients that can be probed
        \begin{itemize}
            \item User-ID agent: client probing
        \end{itemize}
    \item Other NAC mechanisms such as wireless controllers, 802.1x devices, Apple Open Directory servers, or proxy servers.
        \begin{itemize}
            \item User-ID agent: syslog monitoring XMLAPI
        \end{itemize}
\end{itemize}

\subsubsection{User-ID Agent Identification Methods}
\begin{itemize}
    \item Server monitoring:
        \begin{itemize}
            \item Domain controller logs each authenticated user:
                \begin{itemize}
                    \item Microsoft ActiveDirectory
                    \item OpenLDAP
                    \item Novell eDirectory
                \end{itemize}
            \item Shared session monitoring
            \item Exchange: Client access server
        \end{itemize}
    \item WMI Probe:
        \begin{itemize}
            \item Probe the client machine's locally maintained login information
        \end{itemize}
    \item Syslog monitoring:
        \begin{itemize}
            \item Listen for authentication syslog messages from network services
        \end{itemize}
\end{itemize}

\subsubsection{User-ID Agent Types}
Device $\rightarrow$User Identification
The User-ID agent comes in two forms: standalone software agent and PAN-OS agent resident on the firewall
\begin{itemize}
    \item The firewall-based agent is included with PAN-OS software and can be configured from \textit{Device $\rightarrow$ User Identification $\rightarrow$ User Mapping}
    \item The software agent is available for download from Palo Alto Networks and can be installed on one or more Windows PCs on the network to obtain user-specific information
    \item A firewall can communicate with both types of agents at the same time.
    \item The PAN-OS agent monitors up to 100 domain controllers for small and midsize deployments such as small remote offices or lab environments
    \item The firewall-based agent also is required for User-ID redistribution
    \item A software agent can monitor up to 100 domain controllers and/or Exchange Servers. Multiple software agents can be deployed to handle larger environments or multiforest domains.
\end{itemize}

\subsubsection{Firewall Agent vs Windows Agent}
\begin{itemize}
    \item PAN-OS agent
        \begin{itemize}
            \item Good for smaller LDAP scenarios
            \item Polls the LDAP server's security log and only pulls the event ID's related to user and IP info.
            \item Uses very little bandwidth
        \end{itemize}
    \item Windows User-ID agent
        \begin{itemize}
            \item Pull ful security log
            \item Quite bandwidth intensive
            \item Try to keep the Windows User-ID agent as close to its monitored servers as possible
            \item This allows the Windows computer to do all of the intensive processing
        \end{itemize}
\end{itemize}

\subsubsection{User-ID Agent Server Monitoring}
The process by which users are identified is:
\begin{itemize}
    \item The firewall enumerate users and groups (through LDAP)
    \item User-ID agent reads the access logs and maps users to IP addresses
    \item User-ID agent updates information on the firewall
\end{itemize}

\subsubsection{AD Security Logs}
One method the User-ID agent uses to map users to IP addresses is to read this Security log and to record the IP address that each user logged in from:
\begin{itemize}
    \item When the User-ID agent first starts up, it will parse the entire log and record all of the user login events.
    \item Afterward, it will check the log constantly to trap new login events.
    \item These mappings are kept for an amount of time equal to the timeout value set in the User-ID agent interface.
\end{itemize}

\subsubsection{Shared Server Session Monitoring}
An additional server-based method to resolve usernames to IP addresses is to consult the current session table on the domain controller:
\begin{itemize}
    \item Users that are connected to resources on the domain controller (e.g., shared folders and printers) have their IP addresses and username stored in the server session table.
    \item The User-ID agent is able to read this table and use it to maintain user-to-IP mappings
    \item The agent will require server operator privileges to read the session table
\end{itemize}
Monitor open server sessions:
\begin{itemize}
    \item Clients who have connected to a shared resource will have their session information stored on the server.
    \item The User-ID agent reads this session table to maintain any known user-to-IP-address mappings.
\end{itemize}

\subsubsection{Exchange CAS Security Logs}
For situations where the users are not signing onto LDAP, User-ID can identify users when they access their email on a Microsoft Exchange Server:
\begin{itemize}
    \item Exchange Server 2007 and 2010 have five roles that can be distributed to one or more Exchange Servers. User-ID requires the client access server (CAS) role.
    \item In Exchange Server 2005, the mailbox/backend server role is required.
    \item The agent maps all users accessing the Exchange Servers, including users accessing their Microsoft Exchange Mailbox through MSRPC or Active Sync
    \item IMAP/POP3 clients are not supported due to limitation in Exchange Event logging.
\end{itemize}

\subsubsection{WMI Client Probing}
\begin{itemize}
    \item The agent switches to active methods if passive methods fail to identify the user
    \item WMI queries can be sent to workstations with WMI enabled to find users.
\end{itemize}

\subsubsection{Syslog Integration}
Syslog filters, which are provided by a content updated (integrated User-ID agent) or configured manually, allow the User-ID agaent to parse and exctract usernames and IP addresses from authentication syslog events generated by the external service, and add the information to the User-ID IP address-to-username mappings maintained by the firewall

\subsubsection{X-Forwarded-For for User-ID}
\begin{itemize}
    \item The source IP of the packets can be IP address of the proxy when users are behind a proxy
    \item If the proxy server puts the client IP address in HTTP heard X-Forwarded-For, the Palo Alto Networks firewall can extract the client IP address and use it for User-ID
    \item Enable in \textit{Device $\rightarrow$ Setup $\rightarrow$ Client-ID $\rightarrow$ X-Forwarded-For Headers}
\end{itemize}

\subsection{Configuring User-ID}
\subsubsection{Configuring User-ID}
\begin{itemize}
    \item Enable User-ID by zone
    \item Map users to groups:
        \begin{itemize}
            \item Create an LDAP Server profile
            \item Add the LDAP Server profile to the User-ID group mapping configuration
            \item Limit which groups are displayed within the Security policy.
        \end{itemize}
    \item Configure IP address-to-user mapping using User-ID agent and/or other methods, depending on your network.
\end{itemize}

\subsubsection{Enable User-ID by Zone}
\begin{itemize}
    \item The firewall needs to enable User-ID on a per-zone basis
    \item Each zone has an Enable User Identification check box that must be checked to activate the feature.
    \item When User-ID is active, zone-specific lists of allowed and ignored IP addresses can be configured. By default, User-ID is applied to all addresses that originate in that zone.
    \item User-ID tracks only users in the source zone of a session. Never enable User-ID for a zone that contains the Internet, or your firewall will attempt to identify every user in the outside world.
\end{itemize}

\subsection{Mapping Users to Groups}
\subsubsection{User-ID and AD Domains}
The firewall performs all of the LDAP connections:
\begin{itemize}
    \item Users and groups names are stored on the firewall in the format of domain\textbackslash name
    \item The firewall can be configured for either a single AD domain or a multidomain AD forest
    \item Each User-ID agent can monitor only users and domain controllers from a single AD domain
    \item The firewall also can communicate to one or more firewalls in one or more domains.
    \item Each forest must have at least one User-ID agent.
    \item If multiple agents are deployed, they all must be able to communicate to the same firewall.
\end{itemize}

\subsubsection{LDAP Server Profile}
A \textit{Server Profile} object is needed to specify which directory servers will be contacted and the order in which they are contacted.
The Base DN field represents the point in the LDAP tree that the firewall will connect to and begin the search for users and groups.
The Bind DN field contains the username credentials that the firewall uses to connect to the LDAP server. Either fully qualified LDAP name or a user principal name.
For the firewall to access a Global Catalog server the LDAP bind must be set to port TCP 3268 or over SSL connection with LDAP serve use port 636.
If you have difficulties identifying your directory base DN, follow these steps:
\begin{itemize}
    \item Open the AD Users and Groups management console on the domain controller.
    \item Select Advanced features in the View menu of the management console.
    \item Select the top domain object and select Properties
    \item Navigate to the Attribute Editor in the properties window and scroll to the distinguished name attribute.
    \item Copy the content of this attribute into the LDAP server configuration Base field in the firewall WebUI.
\end{itemize}

\subsubsection{Create User-ID Group Mapping Filter}
Before a Security policy can be written for groups of users, the relationships between the users and their groups must be established. 
This information is retrieved from an LDAP directory, such as Active Directory or eDirectory:
\begin{itemize}
   \item The firewall or an agent accesses the directory and search for group objects.
    \item Each group object contains a list of user objects that are members
    \item This list is evaluated and becomes the list of users and groups available in the Security Policy and authentication profiles
    \item The only method to retrieve this data is through LDAP queries from the firewall
\end{itemize}

\subsubsection{Filtering Group Sent to Firewall}
Device $\rightarrow$ User Identification $\rightarrow$ Group Mapping Settings
\newline The \textit{Group Include List} can be used to filter which groups from the LDAP servers are displayed in the Firewall Policy interface.

\subsubsection{show user group-mapping state all Command}
To see if everything is working properly with the group mappings, you can issue the CLI command:
\begin{verbatim}
> show user group-mapping state all    
\end{verbatim}
to see all the User and Group information that was retrieved from the LDAP servers.
This is the recommended method of verifying your LDAP configuration.

\subsubsection{Custom Groups Based on LDAP Filters}
Device $\rightarrow$ User Identification $\rightarrow$ Group Mapping Settings
\newline
Definition of custom groups can be quicker than creating new groups or changing existing ones on an LDAP server, and doesn't require an LDAP administrator to intervene.

\subsection{Mapping Users to IP Addresses with User-ID Agent}
\subsubsection{User-ID Agent Types}
Remember that there are two different User-ID agent types;
\begin{itemize}
    \item One that lives natively in the firewall
    \item The Windows User-ID agent that is deployed outside of the firewall and can be loaded on any windows system
\end{itemize}

\subsubsection{Configuring PAN-OS Integrated User-ID Agent}
\begin{enumerate}
    \item Create a user on the domain controller with the required permissions.
        \begin{itemize}
            \item Windows 2003 domain servers:
                \begin{itemize}
                    \item GPO Policy: Manage Auditing and Security Logs
                \end{itemize}
            \item Windows 2008 or newer:
                \begin{itemize}
                    \item Event Log Readers group
                    \item Distributed COM Users group
                \end{itemize}
            \item WMI probing:
                \begin{itemize}
                    \item Domain administrator and server operator or any other account that has the CIMV2 namespace.
                \end{itemize}
        \end{itemize}
    \item Enter the credentials for the domain controller account
    \item Define the address of the domain controller to be monitored
    \item Configure WMI probing (optional)
    \item Enable NetBIOS probing (optional)
\end{enumerate}

\subsubsection{Define the Monitored Domain Controller}
Each User-ID agent must be configured for the servers it needs to monitor.
The agent includes an \textit{autodiscovery} feature that automatically identifies available Microsoft Windows Servers via DNS for event log monitoring
The firewall will discover domain controllers based on the domain name enter in the Domain field of the 
\textit{Device $\rightarrow$ Setup $\rightarrow$ Management tab $\rightarrow$ General Settings}

\subsubsection{Enter WMI Credentials}
The \textit{WMI Authentication} tab must contain the user account information used to access the LDAP server logs and WMI probing.

\subsubsection{Enable WMI/NETBios Probing}
Make sure that the Windows client firewall will allow WMI probing:
\begin{itemize}
    \item Add a remote admin exception to the firewall
    \item Allow port 139
    \item Enable file and printer sharing
\end{itemize}

\subsubsection{Status: Connected}
Once the setup is complete and committed to the firewall configuration, the user mapping tab will show the connection status for all servers that the User-ID agent is monitoring.

\subsubsection{Configuring the User-ID Agent Introduction}
\begin{enumerate}
    \item Download User-ID agent software
    \item Select a Windows PC for the installation
    \item Run the User-ID agent installer
    \item Configure a Windows service account for the User-ID agent:
        \begin{enumerate}
            \item Give local permission to log in as a service.
            \item Give local permission to access the User-ID agent installation directory.
            \item Assign the service account rights to the User-ID agent registry subtree.
        \end{enumerate}
    \item Configure the software User-ID agent.
    \item Configure the Palo Alto Networks firewall to connect to the User-ID agent.
\end{enumerate}

\subsubsection{Download User-ID Agent Software}
The User-ID Agent software can be downloaded from \href{support.paloaltonetworks.com}{support.paloaltonetworks.com}
under software updates. 
The software has to be installed as a user with admin access to the computer that you are installing the agent on.

\subsubsection{Select a Windows Device for the Installation}
\begin{itemize}
    \item Install on the domain controller
    \item Or install on any Windows machine with access to the domain controller (or Exchange Server) and PAN-OS firewall:
        \begin{itemize}
            \item Microsoft Windows XP SP3 or later
            \item 32-bit and 64-bit are both supported
            \item Install the User-ID agent close to the server it will be monitoring to opimized bandwidth usage.
        \end{itemize}
\end{itemize}

\subsubsection{Windows Service Account for the User-ID Agent}
\begin{itemize}
    \item Create a Windows service account to be used by the User-ID agent on the domain controller.
    \item Give the account local permission to the service.
    \item Give the account local permission to the User-ID agent installation directory.
    \item Assign the account rights to the User-ID agent registry subtree
\end{itemize}

\subsubsection{Configure the User-ID Agent}
The software agent also includes an \textit{autodiscovery} feature to find Microsoft Windows servers.
Access to the servers must be configured in the Authorization tab before information can be gathered from these sources.

\subsubsection{Windows Probing Config}
Log reading is a passive method of user mapping and \textit{probing} is an active method:
\begin{itemize}
    \item The User-ID agent sends a probe on a configurable interval to each learned IP address in its list to verify that the same user is still logged in.
    \item The results of the probe are used to update the record on the agent and then are passed on to the firewall.
    \item Each learned IP is probed once per interval period.
    \item WMI probing must be enabled on the client machines for the probe to be successful.
\end{itemize}
Ensure that large environment have a long enough intervall for all IPs to be probed.
For example, in a network with 6,000 users and an interval of 10 minutes, that would require 10 WMI requests per second from each agent.

\subsubsection{Configure the Firewall to Connect to the User-ID Agent}
The firewall must be configured with information for every User-ID agent it will connect to:
\begin{itemize}
    \item Each agent's IP address and listening port must be specified.
    \item The firewall can connect to multiple agents, including agents on other firewalls.
    \item The communication between the firewall and the User-ID agent is transmitted over an encrypted SSL connection.
\end{itemize}
The firewall has specific, nonconfigurable timers for its communication to the agent:
\begin{itemize}
    \item \textbf{2 seconds}: Get the list of new IP address-to user mapping from the agent.
    This list is a delta of new mappings only.
    \item \textbf{2 seconds}: Send the list of unknown IP addresses that were encountered in traffic to the agent.
    \item \textbf{5 seconds}: Get the agent status. This is a heartbeat used to determine the status of each configured agent.
    \item \textbf{1 hour}: Get the full list of IP address-to-user mappings from the agent.
\end{itemize}

\subsubsection{Confirm Connection to the User-ID Agent}
Device $\rightarrow$ User Identification $\rightarrow$ User-ID Agents
\newline The connection can be confirmed within the system logs and the User-ID Agents tab of the firewall.

\subsubsection{Confirm Connection to the User-ID Agent (Cont.)}
This information can also be viewed via the CLI using the command:
\begin{verbatim}
> show user user-id-agent state <DomainAccountName>
\end{verbatim}

\subsubsection{Display User-ID Mappings}
A view of how users and IP addresses are being mapped on the firewall can be helpful.
This information is available only from the CLI.
\begin{itemize}
    \item \begin{verbatim}
        > show user user-id-agent statistics
    \end{verbatim}
    \item \begin{verbatim}
        > show user user-IDs
    \end{verbatim}
    \item \begin{verbatim}
        > show user ip-user-mapping all
    \end{verbatim}
    \item \begin{verbatim}
        > show user ip-user-mapping <ip/netmask>
    \end{verbatim}
\end{itemize}

\subsubsection{Selecting Users and Groups for a Security Policy}
When you use a user or group in a policy, remember that the source address field and the source user field are evaluated with a logical and condition. 
When you select users for a Security policy, these options are available:
\begin{itemize}
    \item \textbf{any}: Matches any value for user
    \item \textbf{pre-logon}: Used with certain GlobalProtect implementations
    \item \textbf{unknown}: Matches traffic where the user could not be identified by User-ID methods
    \item \textbf{select}: Matches a specific user or group identified by User-ID.
\end{itemize}

\subsection{Mapping Users to IP Addresses with Syslog Integration}
\subsubsection{Syslog Integration Overview}
The Microsoft Windows agent and the built-in PAN-OS agent can listen for authentication syslog messages:
\begin{itemize}
    \item These message are from devices to which users must authenticate such as proxy servers, wireless controllers, and NAC switches.
    \item Syslog filters are used to parse and extract usernames and IP addresses from those syslog messages.
    \item The firewall User-ID agent adds this information to the list of user-to IP-address mappings it maintains.
\end{itemize}
\subsubsection{Configuring Syslog Integration: PAN-OS Software}
Up to 50 syslog senders per virtual system, and up to a total of 100 monitored servers, can be configured per firewall.
The admin selects Syslog Sender from the Type drop-down list.
There are user-defined filters and predefined filters:
\begin{itemize}
    \item The user-defined filter types are defined under \textit{Device $\rightarrow$ User Identification $\rightarrow$ User Mapping $\rightarrow$ Palo Alto Networks User ID Agent Setup $\rightarrow$ Syslog Filters}
    \item Regex Identifier and Field Identifier are the types of user-defined filters that can be configured:
        \begin{itemize}
            \item For Regex Identifier, the admin specifies a regular expression used for pattern matching to identify and extract user-to-IP address information.
            \item Field Identifier parsing uses an admin-defined string to match an authentication event, along with prefix and suffix strings used to identify user mapping information in the syslog messages.
        \end{itemize}
\end{itemize}

\subsubsection{Configuring Syslog Integration: User-ID Agent}
\begin{enumerate}
    \item Open the User-ID agent to configure syslog integration.
    \item Configure the port number to be used:
        \begin{itemize}
            \item 514 for UDP
            \item 6514 for SSL
        \end{itemize}
    \item On the Syslog tab, select the \textit{Enable Syslog Service} check box to enable this option through the User-ID agent.
    \item Add either a Regex Identifier or Field Identifier, then click \textit{OK}
    \item The agent needs to be added in PAN-OS software if this addition has not already been done.
    Add the User-ID agent via \textit{Device $\rightarrow$ User Identification $\rightarrow$ User-ID Agents}
\end{enumerate}